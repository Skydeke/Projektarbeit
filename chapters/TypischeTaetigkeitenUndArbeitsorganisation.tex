\selectlanguage{ngerman}

\subsection{Tools}
\subsubsection{Jira}
Als Organisationstool für Tickets benutzten wir \href{https://www.atlassian.com/software/jira}{Jira}. Es erlaubt uns Einen eigenen
'Lebenszyklus' für die Tickets zu erstellen, welche diese durchlaufen müssen. Das hilft dabei unsere Prozesse auch so umzusetzen wie
es mit dem Team abgemacht war. Es ist aber nicht nur ein Tool zur Ticket-Verwaltung, denn es hat auch einige SCRUM-Tool Features wie
das Backlog (s. Abbildung \ref{quivia_backlog}), in dem alle Tickets die in den nächsten Sprints umgesetzt werden sollen priorisiert
aufgelistet werden.
\par
Auch das Sprint-Board (s. Abbildung \ref{quivia_sprint_board}) von Team Quivia ist in Jira zu finden. Dieses wurde von uns benutzt, um
den Status unserer Aufgaben übersichtlich anzuzeigen. Dieses Board ist vollständig von unserer Scrum Master konfiguriert worden.
Nach einigen Wochen hat sich das Team auf die Folgenden Zustände auf dem Scrum-Board:
\begin{xltabular}{\linewidth}{l|X}
    Spaltenname & Beschreibung \\
    \hline
    TO DO & Hier befinden sich alle Stories, die in einem Sprint aufgenommen werden und innerhalb des Sprints bearbeitet werden
    müssen. \\
    \hline
    \endfirsthead
    IN PROGRESS & Sobald der Entwickler mit der Bearbeitung einer Story beginnt, kommt diese in die Spalte In Progress.
    Die Story bleibt in dieser Spalte, solange der Entwickler daran arbeitet. Im Idealfall sollte jeder Entwickler
    immer nur an einer Story aktiv arbeiten, weshalb sich hier nur eine Story des jeweiligen Entwicklers befinden sollte. \\
    \hline
    IN REVIEW & Ist der Entwickler mit der Bearbeitung der Story fertig und hat alle Akzeptanzkriterien umgesetzt, befindet
    sich die Story in der Spalte ''In Review''. Sobald die Story sich in dieser Spalte befindet, wird der zuständige Betreuer
    darüber in Kenntnis gesetzt, dass der Betreuer diese überprüfen kann. Am besten erfolgt dies über die Kommentar-Funktion im 
    Jira Ticket, mit einer zusätzlichen Zuweisung der Story an den entsprechenden
    Betreuer. Das Review des Betreuers erfolgreich war, geht die Story in die Spalte Sprint Review über.
    Andernfalls muss die Story überarbeitet werden und erneut in den Status To Do übergehen. \\
    \hline
    SPRINT REVIEW & Hier befinden sich alle Stories, die fertig bearbeitet und vom Betreuer abgenommen wurden. Diese Stories
    werden im Sprint Review dem Product Owner und den Stakeholdern vorgestellt und das Ergebnis anhand der geforderten ACs präsentiert.  \\
    \hline
    DONE & Hier befinden sich alle Stories, die vom PO während dem Sprint Review abgenommen wurden, diese Stories gelten somit
    als erledigt. Wenn der PO nach dem Sprint Review noch Änderungen zu der Story aufnehmen möchte, wird
    dazu ein extra Bug-Ticket erstellt oder eine neue Story im Backlog angelegt.  \\
    \hline
\end{xltabular}

\par
Im Projekt Quivia haben wir AGIL mit SCRUM gearbeitet, wobei ein essenzieller Teil von SCRUM die Erstellung von Tickets ist.
Jira, als Ticket-Verwaltungstool bietet uns die Möglichkeit unsere Tickets in verschiedene Arten von Ticket zu Gruppieren, z.B.
EPICs, welche größere Features darstellen, die über untergeordnete User-Stories (s. Abbildung \ref{quivia_user_story}) neue Features beschreiben, oder User-Stories,
welche eine kleinere Funktionalität genau beschreiben und eine Liste mit Akzeptanzkriterien beinhaltet, die ein Entwickler nach und
nach beim Implementieren abhaken soll.

\subsubsection{Confluence}
In der modernen Software-Entwicklung ist die Wichtigkeit von guter Dokumentation allgemein bekannt. Bei doubleSlash wird \href{https://www.atlassian.com/software/confluence}{Confluence}
benutzt für die Dokumentation technischer und nicht-technischer Prozesse, Projekte sowie unternehmensrelevanter Informationen, um eine strukturierte und
kollaborative Plattform für das Teilen von Wissen und die Zusammenarbeit im gesamten Team zu gewährleisten. Confluence (s. Abbildung \ref{quivia_confluence}) hat
Integrationsmöglichkeiten mit Jira, Trello, Miro und Figma, Tools, welche entlang des
Softwareentwicklungsprozesses benutzt werden können, um Informationen aufzubereiten. Confluence bietet durch einen Klartext-Orientierten Editor die Möglichkeit,
Informationen mit wenig Aufwand gut strukturiert und organisiert zu präsentieren.

\subsubsection{Insomnia}
Um Quivia's Backend zu Testen ohne das Frontend mitziehen zu müssen benutzen wir \href{https://insomnia.rest/}{Insomnia}. Insomnia ist eine
Open-Source Alternative zu \href{https://www.postman.com/}{Postman}. Insomnia und Postman sind API-Testing- und Design-Plattformen,
die es Entwicklern ermöglichen, HTTP- und RESTful-APIs zu entwerfen, zu testen und zu debuggen. Mit Insomnia können Entwickler
HTTP-Anfragen erstellen, die Ergebnisse anzeigen und die Interaktion mit APIs verwalten. Wir haben uns bei Quivia für Insomnia
entschieden, da Postman in Unternehmen kostenpflichtig ist und in den neusten Versionen nicht mehr möglich ist nur lokal zu arbeiten, denn
dort sollen alle API Informationen nur in der Clouds gespeichert werden, was für manche Kundenprojekte problematisch ist, denn es
gibt APIs die besonderen Schutz benötigen.

\subsubsection{Obsidian.md}
Oftmals kommt es in der Software-Entwicklung vor, dass man auf eine Code-Stelle stößt, die man später nochmal verbessern sollte. 
Um diese nicht aus den Augen zu verlieren, ist ein Notizprogramm hilfreich. \href{https://obsidian.md/}{Obsidian.md} ist ein Markdown Notizprogramm. 
Es kann außerdem mit zahlreichen Plugins erweitert werden um die benötigten Funktionen abzudecken. Es gibt zum Beispiel ein Plugin, 
das es ermöglicht alle Notizen und Anhänge über das Versionskontrollsystem Git zu verwalten.

\subsection{Frameworks}
\subsubsection{Flask: CPO Data Portal}
Für das CPO Data Portal, welches später in Kapitel \ref{cpo_data_portal} noch einmal genauer erklärt wird, benutzen wir Python \cite{Python}, mit dem Flask-Framework \cite{Flask}.
Das Flask-Framework ermöglicht Entwicklern mit wenig Aufwand eine Web-Applikation zu erstellen. Es wird oft auch als Microframework bezeichnet, da 
es sich nicht auf bestimmte Libraries oder Tools festlegt, und nur einen minimalistischen Webserver mitliefert. Das bedeutet, dass ein Entwickler
sich selbst um kompliziertere Themen wie eine Abstraktionsschicht für eine Datenbank oder Ähnliches kommen muss. Die Idee dahinter ist, dass nur
die Libraries hinzugefügt werden die auch wirklich benötigt werden, was ein modularer und maßgeschneiderten Ansatz verspricht.

\subsubsection{Spring Boot: Quivia}
Spring Boot \cite{SpringBoot} ist ein Java-Framework das Entwickeln von robusten, skalierbaren und wartbaren Backends. Es sticht heraus durch 
die Möglichkeit wenig selbst zu konfigurieren, da Spring Boot automatisch viele der Konfigurationen für uns erledigt. Ein weiterer Vorteil der 
Nutzung des Spring-Frameworks ist, dass viele der Libraries, welche wir benötigen, eine gute Integration mit Spring haben. Quivia nutzt 
Spring-Security um den Zugriff auf Rest-Endpunkte zu verwalten. Es hat auch Integrationsmöglichkeiten mit Systemfunktionen, wie z.B. 
Cron \cite{Cronjob}. Da Quivia auch auf Microsoft's Azure Cloud deployt ist, erlaubt uns die Nutzung von diesen Spring Features Busy-Waiting zu 
verhindern.
