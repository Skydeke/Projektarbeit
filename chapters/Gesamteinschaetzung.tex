\selectlanguage{ngerman}
Ich habe während meinem Praxissemester viele der Konzepte und Dinge aus den Fächern wiederverwenden können. Unten habe ich kurz tabellarisch 
aufgeführt, was ich alles gelernt und vertieft habe. Natürlich ist diese Liste nicht vollständig und Fächer wie 'Betriebsysteme' waren sehr 
hilfreich, denn ich hatte unter anderem ein Problem das mit einer Message - Queue zusammen hing, was ich nur lösen konnte, weil 
dieses Hintergrundwissen vorhanden war. 

\begin{xltabular}{\linewidth}{l|c|c|X}
  Kompetenz & Vertieft                                 & Neu gelernt       & Kommentar \\
  \hline
  Java & x & - & \\ 
  \hline
  Python & x & - & \\ 
  \hline
  SCRUM & x & x & SCRUM im generellen wurde in Studium ausführlich besprochen und auch teilweise eingesetzt, aber es waren auch Dinge dabei, die man erst beim Anwenden lernt, z.B. Umgehen mit Sprint-Review Stresssituation \\
  \hline
  Docker & x & - & Docker war sehr wichtig für lokale Entwicklung mit dem CosmosDB Emulator und für deployment in die Azure Cloud \\
  \hline
  OOP & x & - & Die OOP war grundlegend für das Lösen vieler Probleme \\
  \hline
  Dokumentendatenbanken & - & x & Dokumenten - Datenbanken als solche wurde nicht im Studium erklärt, das Umdenken auf eine nicht-relationale Weise war eine Herausforderung \\
  \hline
  Spring-Boot & - & x & Quivia \\
  \hline
  VueJs & x & - & Quivia \\
  \hline 
  Flask & - & x & CPO Data Portal \\ 
  \hline
  HTML, CSS, JS & x & - & Quivia \\ 
  \hline
  Dokumentation schreiben & x & - & AsciiDoc, Confluence, OpenAPI, AsyncApi \\ 
  \hline 
  Tests schreiben & x & - & JUNIT 5 \\
  \hline
\end{xltabular}

\newpage

Auch die Betreuung durch meine Betreuer Marc und Johannes war super. Ich hatte das Glück 
Marc jederzeit bei Problemen oder Ideen zur Lösung eines Problems um Hilfe bitten zu dürfen. Johannes 
hatte sich um meine Code - Reviews gekümmert und hat sich auch mehrfach als Stakeholder in 
Sprint - Reviews eingebracht.

Abschließend kann ich das Praktikum bei doubleSlash nur weiterempfehlen, das Arbeitsklima ist super, 
genauso wie die Betreuung. Anfangs dauert es etwas bis man sich an die ganzen Meetings und die 
Organisation um das Arbeiten gewöhnt hat, aber sobald man dann mal 'angekommen' ist, macht es auch richtig Spaß. 
Es hat mir tatsächlich so viel Spaß gemacht, das ich nahezu nahtlos als Werkstudent weitergemacht habe.
