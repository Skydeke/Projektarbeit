\selectlanguage{ngerman}

\subsection{Einleitung}
Bei dieser Projektarbeit soll ein vollwertiges Linux für ein Embedded-Systems-Gerät
mit beschränkten Resourcen gebaut und deployt werden. Es soll aber nicht nur ein
Linux mit Terminal (TTY) sein, sondern es soll möglich sein eine volle Grafische
Bedienoberfläche darzustellen. Es ist Ziel, dass ein Browser, der eine Landing-Page
anzeigt, möglichst schnell nach Einschalten des Geräts angezeigt wird.

\subsection{Anforderungen an Hardware}
Ein echtes vollwertiges Linux auf einem Micro-Controller zu starten ist nicht möglich,
denn ein vollwertiges Linux benötigt eine Memory-Management-Unit (MMU). Eine MMU
erlaubt dem Betriebssystem einen virtuellen Adressraum zu benutzen. Das hat zu Folge,
dass Prozess - Isolierung aufgeweicht und ''Swapping'' - einen Bereich auf der
Festplatte auf den Prozesse ausgelagert werden, um den RAM freizuhalten,
unmöglich ist. Linux unterstützt seit Kernel 2.5.46 offiziell Micro-Controller ohne
MMU \cite{uCLinuxWikipedia}.

Es sollten auch folgende Resources zur Verfügung stehen:
\\ - 2 MB Flash für einen Bootleader
\\ - min. 8 MB RAM
\\ - genug Speicher für ein Root-Filesystem, auf dem alle Programme installiert werden

Für diese Projektarbeit wurde das STM32F469I-Board gewählt \cite{stmWebsiteBoard}.
Dieses Board erfüllt die Requirements für Linux und hat zusätzlich ein Display welches
über Digital Serial Interface (DSI) angesprochen wird.
