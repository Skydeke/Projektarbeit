\selectlanguage{ngerman}

\subsection{uClibc-ng}
Die Standard-C-Library gegen die Kompiliert wird, hat Einfluss darauf wie sich das
kompilierte Programm verhält. Für Microprozessoren und Systeme mit begrenzten Resourcen,
gibt es eine alternative Implementierung des Standard - Library, die speziell für das
Bauen von Linux ohne MMU - Support geschrieben wurde.

\subsection{U-Boot}
''Das U-Boot'' ist ein Open-Source Bootloader, der für Embedded Systeme ausgelegt
wurde \cite{uBootSourceTree}. Besonders an diesem Bootloader ist die Möglichkeit viele
der Features zur Compile-Time zu deaktivieren, was kleinere Kompilate zur folge hat.

\subsection{Buildroot}
Buildroot ist ein Framework mit denen man ein Linux mit Root-Filesystem für
Systeme mit beschränkten Resourcen zu bauen. Das Root-Dateisystem kann auch mit
den verschiedenen Tools von Busybox gefüllt werden, wodurch die Betriebssysteminstallation
ein vollwertiges GNU/Linux wird.

\subsection{Busybox}
Busybox ist eine Sammlung minimaler Tools, die oft in POSIX - Systemen verwendet werden.
Es ist eine einzige ausführbare Binärdatei, die über 300 UNIX-Tools bereitstellt.
Darunter sind bekannte Befehle wie ls, cat und grep.

