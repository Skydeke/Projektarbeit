\selectlanguage{ngerman}

\section{Optimierung von Init-Skripten und Systemstart}

\subsection{Methoden zur Optimierung}
Um den Systemstart zu beschleunigen, gibt es mehrere Ansätze. Ein wichtiger Punkt ist, die Anwendung so früh
wie möglich zu starten, damit sie nicht unnötig aufgeschoben wird. Auch das Vereinfachen von Shell-Skripten
trägt zur Optimierung bei, da komplexe oder ineffiziente Skripte den Bootprozess verlangsamen.

\subsection{Messung der Bootzeit}
Um die Auswirkungen solcher Optimierungen zu überprüfen, gibt es verschiedene Werkzeuge. \textit{bootchartd}
zeichnet während des Init-Prozesses alle gestarteten Prozesse auf. Die gesammelten Daten lassen sich dann mit
\textit{bootchart.jar} grafisch auswerten. Für Systeme mit \textit{systemd} bietet \textit{systemd-analyze}
eine detaillierte Analyse des Bootvorgangs und zeigt unter anderem die Startzeiten einzelner Dienste an.
Allerdings ist \textit{systemd} nicht ideal für Systeme mit beschränkten Ressourcen, da es verhältnismäßig
viel Speicher und CPU-Leistung benötigt. Leichtgewichtigere Alternativen wie \textit{BusyBox init} sind in
solchen Fällen oft besser geeignet.

Da das STM32F769I stark Beschränkte Resourcen hat, wird BusyBox init als Init-System verwendet.

\subsection{Init-Skripte}
Auch bei den Init-Skripten lässt sich einiges an Ressourcen sparen. Ein bewährter Trick besteht darin, alle
Dienste über ein einziges Startskript aufzurufen. So vermeidet man wiederholte Starts von \textit{/bin/sh},
was Systemressourcen schont. Zusätzlich lohnt es sich, wann immer möglich auf Shell-Builtins zurückzugreifen,
statt externe Programme zu starten – das spart \textit{fork/exec}-Aufrufe und reduziert den Overhead. Pipes
und Backticks sollte man ebenfalls sparsam einsetzen, da sie oft zusätzliche Prozesse erzeugen und den Start
dadurch verlangsamen.

Zum Testen des Bildschirms nutze ich bei meinem Testgerät den in Linux integrierten Video Pattern Generator
(VPG). Das Init-Skript lauscht dabei auf den USR-Button: Mit jedem Tastendruck wird durch die VPG-Modi
geschaltet, bis der Generator am Ende wieder deaktiviert wird.

\subsection{Schneller Splashscreen}
Ein schneller Bootprozess wirkt oft noch flüssiger, wenn ein passender Splashscreen angezeigt wird. Eine
einfache Möglichkeit dafür ist \textit{fbsplash}, das schnell ein Startbild anzeigen kann. Wer es noch
schneller haben will, kann das Bild direkt aus dem Framebuffer auslesen, mit \textit{lzop} komprimieren und
das komprimierte Bild ins \textit{initramfs} legen. Für animierte Splashscreens bietet sich ein kleines
C-Programm an, das direkt auf den Framebuffer zeichnet – das geht schnell und kommt ohne externe
Abhängigkeiten aus. Am schnellsten ist es allerdings, wenn der Bootloader selbst bereits ein Bild anzeigt.
U-Boot unterstützt diese Funktion und kann ein Splashscreen direkt beim Start einblenden.

