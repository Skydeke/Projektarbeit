\selectlanguage{ngerman}

\section{Grundlagen der Bootzeitoptimierung}
Um die Bootzeit gezielt zu optimieren, muss sie präzise gemessen werden. Wichtige Referenzpunkte sind der
Start des Bootloaders, Endzeit des Bootloaders (wenn der Kernel geladen wird), Dekompressionszeit des Kernels
Startzeit der Services, initialisierung der Hardware und die Userscript-Ausführungszeit.

\subsection{Messmethoden}
Bootlin empfiehlt die folgenden Messmethoden:
\begin{itemize}
	\item \textbf{Oszilloskop:} Eignet sich gut, um präzise Messungen direkt an den GPIO-Pins oder der Stromversorgung durchzuführen.
	\item \textbf{Mikrocontroller (z.B. Arduino):} Eine günstige Möglichkeit, die Spannung zu überwachen.
	\item \textbf{Serielle Konsole:} Ermöglicht das Aufzeichnen von Boot-Zeitstempeln mit \textit{grabserial}.
\end{itemize}

Wir verwenden \textit{grabserial}, da es einfach einzurichten ist und ohne zusätzliche Hardware auskommt. Es
ermöglicht uns eine automatisierte und reproduzierbare Erfassung der Bootzeiten. Da die Messungsergebnisse nur
aussagekräftig sind wenn sie zwischen den den Messungen eine geringe abweichung haben, werden alle
Konfigurationen 10x gemessen, und nur als Verbesserung angenommen wenn sie wenig voneinander abweichen.

\subsection{Bootphasen}
Ein typischer Bootvorgang beginnt mit dem ROM-Code, der das Bootstrap-Image in den SRAM lädt. Anschließend
übernimmt der X-Loader bzw. U-Boot SPL, initialisiert den DRAM und lädt den eigentlichen Bootloader. Der
U-Boot läd das Kernel-Image, welches entpackt wird und startet dann das Kernel-Image. Sobald der Linux-Kernel
die Kontrolle übernimmt, werden die Hardware initialisiert und erste Systemdienste gestartet. Schließlich
beginnt der Userspace, in dem Startup-Skripte ausgeführt und alle notwendigen Prozesse für den
Betriebssystembetrieb gestartet werden.

\subsection{Optimierungsreihenfolge}
Es wird empfohlen, mit den Startup-Skripten zu beginnen und unnötige Dienste zu deaktivieren. Danach sollte
der Kernel optimiert werden, indem nur benötigte Module geladen werden. Auch sollten die verschiedenen
Kernel-Kompressionsalgorithmen gemenchmarkt werden. Abschließend lassen sich Verzögerungen im Bootloader
reduzieren und die Ladezeiten optimieren.
