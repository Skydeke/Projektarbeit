\selectlanguage{ngerman}

\section{Filesystem Optimierungen}

\subsection{Filesystem Auswahl}
Die Wahl des Dateisystems beeinflusst die Bootzeit erheblich. Grundsätzlich lassen sich Dateisysteme in zwei
Kategorien einteilen: Block-Storage-Dateisysteme wie \textit{ext2/4}, \textit{xfs}, \textit{btrfs},
\textit{f2fs}, \textit{SquashFS} und \textit{EROFS} sowie Raw-Flash-Dateisysteme wie \textit{JFFS2},
\textit{YAFFS2} und \textit{UBIFS}. Aufgrund der begrenzten Speicherkapazität des genutzten Flash-Speichers
wurden in den Tests jedoch ausschließlich Block-Storage-Dateisysteme untersucht.

Für Read-Only-Root-Dateisysteme hat sich \textit{SquashFS} nach den Benchmarks von Bootlin als eine gute Wahl
erwiesen, da es schnelle Mountzeiten und hohe Lesegeschwindigkeit bietet. \textit{EROFS} ist eine neuere
Alternative, die ebenfalls für Read-Only-Dateisysteme optimiert wurde und eine noch bessere
Lesegeschwindigkeit erreichen kann. Falls Schreibzugriffe erforderlich sind, bietet sich \textit{UBIFS} mit
aktiviertem \textit{UBI Fastmap} an, um die Mountzeiten weiter zu verbessern.

\subsection{Initramfs}
Ein kleines \textit{initramfs} kann den Bootvorgang beschleunigen, indem es als Zwischenschritt vor dem
eigentlichen Root-Dateisystem dient. Es kann entweder direkt in das Kernel-Image integriert oder vom
Bootloader geladen werden. Besonders für kleine Root-Dateisysteme ist dies vorteilhaft. Die Erstellung eines
\textit{initramfs} erfolgt durch das Erstellen eines CPIO-Archivs, das anschließend komprimiert wird.
Alternativ kann beim Kernel-Kompilieren die Option \textit{CONFIG\_INITRAMFS\_SOURCE} genutzt werden, um das
\textit{initramfs} direkt in das Kernel-Image zu integrieren. Um die Größe gering zu halten, empfiehlt es
sich, statisch gelinkte Anwendungen zu verwenden, was in Buildroot über \textit{BR2\_STATIC\_LIBS}
konfiguriert werden kann. Zudem sollte vermieden werden, ein komprimiertes \textit{initramfs} innerhalb eines
bereits komprimierten Kernel-Images zu verwenden (\textit{CONFIG\_INITRAMFS\_COMPRESSION\_NONE}), um doppelte
Komprimierung zu verhindern.

\subsection{Filesystem Optimierung Ergebnisse}
Tests auf dem STM32F769I zeigten, dass \textit{ext2} in Kombination mit einem \textit{initramfs} die besten
Bootzeiten liefert. Andere Block-Storage-Dateisysteme waren entweder langsamer oder boten keinen nennenswerten
Vorteil für den Einsatzzweck. Aufgrund begrenzter Flash-Kapazitäten wurden Raw-Flash-Dateisysteme nicht
getestet.

